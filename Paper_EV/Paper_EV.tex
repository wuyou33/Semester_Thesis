% This is one possible realization of the official FSD Microsoft Word template. The title page has to be filled in in Microsoft Word environment and saved as a PDF file. Note, that in the header and footer the name of the author and the thesis topic have to be provided. 
%
% Compile Template in following order:
%
% pdflatex Paper_EV
% makeindex Paper_EV.nlo -s nomencl.ist -o Paper_EV.nls (Table of Symbols)
% makeindex Paper_EV.acn -s Paper_EV.ist -t Paper_EV.alg -o Paper_EV.acr (Table of Acronyms)
% makeindex Paper_EV.glo -s Paper_EV.ist -t Paper_EV.glg -o Paper_EV.gls
% pdflatex Paper_EV
% pdflatex Paper_EV
%
% TODO:
% - Title page: Dr. -> Dr. ir.
% - Acronyms, right order of full and short appearences
% - Create List of Symbols
%
%#################################################################################
%#################################################################################
%#################################################################################

% Preamble
\documentclass[11pt, a4paper, twoside]{report}

% Define page layout
\usepackage[
 left=2.5cm,
 right=2.5cm,
 top=2.5cm,
 bottom=3cm]
{geometry}

% Package for defining the default font size
% We use Helvetica as an alternative to Arial
\usepackage{helvet}
\renewcommand{\familydefault}{\sfdefault}
%alternatively one can use Arial package and compile with XeLaTEX
%\usepackage{fontspec}
%\setmainfont{Arial}

% For inserting images, figures
\usepackage{tikz}

% For citation of URLs
\usepackage{url}

% Define subsection depth
\setcounter{secnumdepth}{3}

% Set line spacing to 1.15pt
\renewcommand{\baselinestretch}{1.15}

% Set spacing between paragraphs to be 10pt
\setlength{\parskip}{10pt}

% Set paragraph intendation (here 0pt, no intendation)
\setlength{\parindent}{0pt}

% Package for estimating the last page
\usepackage{pageslts}

% Package for bold symbols and letters
\usepackage{bm}

% Package for equations and math. symbols
\usepackage{amsmath}
\usepackage{amssymb}

% Package and settings for adding header/footer
\usepackage{fancyhdr}

% Set style for pages with arabic numbering 
% This block prints only the chapter number and the chapter name, not the word "Chapter"
%\renewcommand{\chaptermark}[1]{\markboth{\MakeUppercase{\ \thechapter. \ #1}}{}}
% Clear current page style 
\fancyhf{}
% Head (11 pt font size and 12 pt line spacing)
\fancyhf[HRE,HLO]{\includegraphics[width=2.5cm]{figs/fsd}}
\fancyhf[HRO,HLE]{\fontsize{11}{12}\selectfont \leftmark}
% Foot (11 pt font size and 12 pt line spacing)
\fancyhf[FRE,FLO]{\fontsize{11}{12}\selectfont Design, Implem. and Eval. of an INDI Controller for a Nano-Quadrotor\\\vspace*{0.1cm}Evghenii Volodscoi}
\fancyhf[FLE,FRO]{\fontsize{11}{12}\selectfont Page \thepage \thinspace / \lastpageref{LastPages}}
% Line settings
\renewcommand{\headrulewidth}{0.1pt}	%upper line
\renewcommand{\footrulewidth}{0.1pt}	%lower line

% Set style for pages with roman numbering 
\fancypagestyle{fancy_beginning}{%
%\pagestyle{fancy}
% Clear current page style 
\fancyhf{}%
% Head
\fancyhf[HRE,HLO]{\includegraphics[width=2.5cm]{figs/fsd}}%
\fancyhf[HRO,HLE]{\fontsize{11}{12}\selectfont \leftmark}%
% Foot
\fancyhf[FRE,FLO]{\fontsize{11}{12}\selectfont Design, Implem. and Eval. of an INDI Controller for a Nano-Quadrotor\\\vspace*{0.1cm}Evghenii Volodscoi}%
\fancyhf[FLE,FRO]{\fontsize{11}{12}\selectfont Page \thepage}%
% Line settings
\renewcommand{\headrulewidth}{0.1pt}%	%upper line
\renewcommand{\footrulewidth}{0.1pt}%	%lower line
}

% Package for inserting titel page
\usepackage{pdfpages}

% Package for changing distance between sections
\usepackage{titlesec} 

% Package for setting up the caption options of figures
\usepackage{subcaption}

% Package prevents placing floats before a section (for not mooving figures)
\usepackage{float}

% TODO
% Package for using acronyms 
%\usepackage[printonlyused]{acronym}
%\renewcommand\acsfont{\textnormal}	% Set non-bold font of the acronym units

% This patch removes the insertion of \addvspace to both the LoF and the LoT; the cause of the additional gap between entries on a per-chapter basis
\usepackage{etoolbox}% http://ctan.org/pkg/etoolbox
\makeatletter
% \patchcmd{<cmd>}{<search>}{<replace>}{<succes>}{<failure>}
\patchcmd{\@chapter}{\addtocontents{lof}{\protect\addvspace{10\p@}}}{}{}{}% LoF
\patchcmd{\@chapter}{\addtocontents{lot}{\protect\addvspace{10\p@}}}{}{}{}% LoT
\makeatother

% Set chapter font size and spacing
\newcommand{\chapfnt}{\fontsize{14}{19}}
\titleformat{\chapter}
{\normalfont\chapfnt\bfseries}{\thechapter}{1em}{\chapfnt} 	% set font size
\titlespacing{\chapter}{0cm}{-7mm}{0cm}					% set spacing

% Set section font size (13pt) and spacing
\newcommand{\secfnt}{\fontsize{13}{19}}
\titleformat{\section}
{\normalfont\secfnt\bfseries}{\thesection}{1em}{}
\titlespacing*{\section}{0cm}{0.3cm}{0cm}					% set spacing

% Package for edditing TOC, LOF and LOT
\usepackage[titles]{tocloft}

% Package for setting itemization/description parameters like indentation of items
\usepackage{enumitem}

% Package for acronyms
\usepackage[acronym, automake, nopostdot, nogroupskip]{glossaries}
\setlength{\LTleft}{-0.15cm}			% left justified
\renewcommand*{\arraystretch}{1.4}		% vertical distance between entries, default is 1
% List
\newacronym{EOM}{EOM}{Equations of Motion}
\newacronym{IMU}{IMU}{Inertial Measurement Unit}
\newacronym{INDI}{INDI}{Incremental Nonlinear Dynamic Inversion}
\newacronym{MCU}{MCU}{Microcontroller}
\newacronym{NED}{NED}{North East Down}
\newacronym{NDI}{NDI}{Nonlinear Dynamic Inversion}
\newacronym{ToF}{ToF}{Time-of-Flight}
% Make gls
\renewcommand*\entryname{Acronym}
\makeglossaries

% Package for setting up the list of symbols
%https://tex.stackexchange.com/questions/112884/how-to-achieve-nomenclature-entries-like-symbol-description-dimension-and-uni
\usepackage{nomencl,etoolbox,ragged2e,siunitx}
%
\newcommand{\DimensUnits}[2]{\makebox[1em]{#1}\makebox[4em]{#2\hfill}\ignorespaces}
\newcommand{\insertnomheaders}{\item[\bfseries Symbol]\DimensUnits{}{\textbf{Unit}}\textbf{Description}}
%
\renewcommand\nomgroup[1]{%
  \item[\large\bfseries
  \ifstrequal{#1}{A}{Acronyms}{%
  \ifstrequal{#1}{B}{Latin Letters}{%
  \ifstrequal{#1}{G}{Greek Letters}{%
  \ifstrequal{#1}{S}{Indices}{% 
  \ifstrequal{#1}{U}{Subscripts}{%   
  \ifstrequal{#1}{X}{Other Symbols}{}}}}}}]
  \insertnomheaders
  }
%
\renewcommand*{\nompreamble}{\markboth{\nomname}{\nomname}}
% parbox - Description
\newcommand{\nomdescr}[1]{\parbox[t]{10cm}{\RaggedRight #1}}
\newcommand{\nomwithdim}[5]{\nomenclature[#1]{#2}{\DimensUnits{}{#5}\nomdescr{#3}}}
%
\makenomenclature


%#################################################################################%#################################################################################%#################################################################################

\begin{document}

% Activate roman page numbering
\pagenumbering{roman}

% Insert title page
\includepdf[pages={1}]{figs/Title_page.pdf}

%Insert empty page after title page
\newpage\null\thispagestyle{empty}\newpage


% For some reason fisrt chapter has a different vert. spacing from the top, so we add additional 0.8cm, starred command makes sure it will not be suppressed at the beginning or at the end of the page
\vspace*{-0.8cm}
\section*{Statutory Declaration}
I, Evghenii Volodscoi, declare on oath towards the Institute of Flight System Dynamics of Technische Universität München, that I have prepared the present Semester Thesis independently and with the aid of nothing but the resources listed in the bibliography.
\\
\\
This thesis has neither as-is nor similarly been submitted to any other university.
\bigskip
\\
\\
Garching, 

% Manually add Chapter names to the header (per default not included by Latex)
\thispagestyle{fancy_beginning}
\renewcommand{\chaptermark}[1]{\markboth{#1}{}}
\chaptermark{Statutory Declaration}

%Insert empty page after statutory declaration section
\newpage\null\thispagestyle{empty}\newpage

\section*{Kurzfassung} \label{sec:kurzfassung}
% Start cursive block
\begin{itshape}
Deutsche Kurzfassung der Arbeit.
\end{itshape}


% Add Abstract chapter to the same page as Kurzfassung
{\let\clearpage\relax\section*{Abstract}}
% Start cursive block
\begin{itshape}
English abstract of the thesis.
\end{itshape}

% Manually add Chapter names to the header (per default not included by Latex)
\thispagestyle{fancy_beginning}
\renewcommand{\chaptermark}[1]{\markboth{#1}{}}
\chaptermark{Abstract}

%Insert empty page after abstract page
\newpage\null\thispagestyle{empty}\newpage



% Add Table of Contents 
\setlength{\cftbeforechapskip}{0.2cm}			% space before chapters
\renewcommand\cftchapafterpnum{\vskip-0.4cm}	% space after chapters
\renewcommand\cftsecafterpnum{\vskip-0.4cm}		% space after sections
\renewcommand\cftsubsecafterpnum{\vskip-0.4cm}	% space after subsections	
\renewcommand{\cftchapfont}{\normalfont}		% set normal font, not bold
%\titleformat{\section}{\fontsize{14}{19}}	 	% ?
 
\renewcommand*\contentsname{Table of Contents}  % Change the defaulf name "Contents" to "Table of Contents"  
\tableofcontents								% Generate Table of Contents

% Manually add Chapter names to the header (per default not included by Latex)
\thispagestyle{fancy_beginning}
\renewcommand{\chaptermark}[1]{\markboth{#1}{}}
\chaptermark{Table of Contents}

%Insert empty page after table of contents
\newpage\null\thispagestyle{empty}\newpage

% Add List of Figures  
\renewcommand{\cftfigfont}{Figure }				% Add "Figure" to the LOF
\renewcommand\cftfigaftersnum{:} 				% Put ":" after the figure num 
\setlength{\cftfigindent}{0pt}					% Remove left indent 
\setlength\cftbeforefigskip{-0.2cm}				% Space between figures 
\addtocontents{lof}{\vspace*{10pt}}			    % Set distance between LOF title and list
\listoffigures
%\setlength{\cftbeforeloftitleskip}{0cm}		% ? 
%\setlength\cftafterloftitleskip{-2cm}


% Manually add Chapter names to the header (per default not included by Latex)
\thispagestyle{fancy_beginning}
\renewcommand{\chaptermark}[1]{\markboth{#1}{}}
\chaptermark{List of Figures}

%Insert empty page after table the list of figures
\newpage\null\thispagestyle{empty}\newpage

% Add List of Tables
\renewcommand{\cfttabfont}{Table }				% Add "Figure" to the LOF
\renewcommand\cfttabaftersnum{:} 				% Put ":" after the figure num 
\setlength{\cfttabindent}{0pt}					% Remove left indent
\setlength\cftbeforetabskip{-0.2cm}				% Space between tables
\addtocontents{lot}{\vspace*{10pt}}			    % Set distance between LOT title and list
\listoftables{}

% Manually add Chapter names to the header (per default not included by Latex)
\thispagestyle{fancy_beginning}
\renewcommand{\chaptermark}[1]{\markboth{#1}{}}
\chaptermark{List of Tables}

%Insert empty page after the list of tables 
\newpage\null\thispagestyle{empty}\newpage

% Print table of acronyms
\printglossary[type=\acronymtype, title={Table of Acronyms}, nonumberlist, style=longheader]

% Manually add Chapter names to the header (per default not included by Latex)
\thispagestyle{fancy_beginning}
\renewcommand{\chaptermark}[1]{\markboth{#1}{}}
\chaptermark{Table of Acronyms}

%Insert empty page after table of acronyms
\newpage\null\thispagestyle{empty}

% Add Table of Symbols
%\chapter*{Table of Symbols} \label{sec:symbols}
\mbox{}
\nomwithdim{B}{\( F \)}{Force}{L}{$N$}
\nomwithdim{B}{\( g \)}{Gravitational acceleration}{}{$m/s^2$}
\nomwithdim{G}{\( \alpha \)}{Angle of attack}{}{$rad$}
\nomwithdim{G}{\( \zeta \)}{Damping of a linear second order system}{}{--}
\nomwithdim{S}{\( m \)}{Variable related to pitch moment}{}{}
\nomwithdim{S}{\( W \)}{Wind}{}{}
\renewcommand{\nomname}{Table of Symbols}
\printnomenclature[6em]

% Manually add Chapter names to the header (per default not included by Latex)
\thispagestyle{fancy_beginning}
\renewcommand{\chaptermark}[1]{\markboth{#1}{}}
\chaptermark{Table of Symbols}

%Insert empty page after table of symbols
\newpage\null\thispagestyle{empty}\newpage

% Beginning of the core text 
%%%%%%%%%%%%%%%%%%%%%%%%%%%%%%%%%%%%%%%%%%%%%%%%%%%%%%%%%%%%%%%%%%%%%%%

% Activate arabic page numbering from here on
\pagenumbering{arabic}

% Delete the word "Chapter" leaving only the number of the chapter
\titleformat{\chapter}
{\normalfont\chapfnt\bfseries}{\thechapter}{1em}{\chapfnt} 	% set font size
\titlespacing*{\chapter}{0cm}{-7mm}{0cm}					% set spacing

% Set subsection font size (11pt) and spacing
\newcommand{\ssecfnt}{\fontsize{11}{14}}
\titleformat{\subsection}
{\normalfont\ssecfnt\bfseries}{\thesubsection}{1em}{}
\titlespacing*{\subsection}{0cm}{-0cm}{0cm}					% set spacing

% Activate default fancy design (chapter number, chapter)
\pagestyle{fancy} 
\renewcommand{\chaptermark}[1]{\markboth{\ \thechapter \ #1}{}}

% Change separator in the figure and table captures from "." to "-"
\renewcommand{\thefigure}{\thechapter-\arabic{figure}}
\renewcommand{\thetable}{\thechapter-\arabic{table}}



\chapter{Introduction} \label{cha:introduction}

% Manually add Chapter names to the header (per default not included by Latex)
\thispagestyle{fancy}
\chaptermark{Introduction}

\section{Motivation} \label{sec:motivation}

\section{Contribution of the Thesis} \label{sec:contribution_ofthe_thesis}

- 3. Implementation in Simulink for testing (both loops)\\
- 3. Implementation on HW (outer loop)  

\section{Structure of the Thesis} \label{sec:structure_ofthe_thesis} 



\chapter{Theoretical Background} \label{cha:theoretical_background}

The aim of this chapter is to provide the theoretical background which serves as a basis for some of the methods which are presented and applied in the course of this thesis. At the beginning, in section \ref{sec:eqs_motion} general \acrfull{EOM} of an aircraft are presented. These are later used to derive the control algorithm and build the Matlab/Simulink model of the Crazyflie quadrotor. Sections \ref{sec:ndi} and \ref{sec:indi} describe the \acrfull{NDI} and \acrfull{INDI} methods in general. Later in this chapter (subsections \ref{subsec: indi_inner} and \ref{subsec: indi_outer}) these methods are used to derive the inner and the outer loop of the \acrshort{INDI} flight controller for the Crazyflie quadrotor.

% Manually add Chapter names to the header (per default not included by Latex)
\thispagestyle{fancy}
\chaptermark{Theoretical Background}

\section{Dynamic Equations of Motion of an Aircraft} \label{sec:eqs_motion}

The dynamic equations which describe the general motion of an aircraft are usually coupled first order implicit nonlinear differential equations. However, considering a quadrotor control problem a lot of plausible assumptions (e.g. neglecting the Coriolis and centrifugal acceleration due to the earth rotation) can be made to obtain more simplified versions of those equations. This section presents simplified general \acrshort{EOM} and describes the assumptions which were considered to derive them.

For the derivation of the equations of motion the aircraft system will be assumed to be a rigid body. Such a rigid body system can be described uniquely by 12 states. Note that by taking in account additional effects, such as propulsion system dynamics, multi-body dynamics etc., the number of required state variables will increase. 

Additionally to the rigid body assumption, it is also assumed that the earth is flat and non-rotating and the reference point of the sum of all external forces acting on the body corresponds to the center of gravity of the body. Thus, the linear momentum equation is written as
\begin{equation}
	\begin{split}
		\sum (\bm{F}^G)_B = m \cdot \Big[ (\bm{\dot{v}}^G)_{B} + (\bm{\omega}^{OB})\times(\bm{v}^R)_{B} \Big]
		\label{eq:lin_momentum}
	\end{split}
\end{equation}
where $\sum(\bm{F}^G)_B \in \mathbb{R}^{3 \times 1}$ is the sum of the external forces acting on the system and applied to the center of gravity $G$, $m$ the mass of the body, $(\bm{\dot{v}}^G)_{B} \in \mathbb{R}^{3 \times 1}$ the linear acceleration of the point $G$, $(\bm{\omega}^{OB}) \in \mathbb{R}^{3 \times 1}$ the angular velocity of the body-fixed frame ($B$) with respect to the \acrfull{NED} ($O$) coordinate frame. Subscript $B$ denotes that all variables are specified in the body-fixed coordinate frame.

The rotational motion of a body is described with the angular momentum equation. To derive it, additionally to the assumptions made above it is also considered that the mass and the mass distribution are quasistationary, meaning $\frac{d}{dt}m=0$ and $\frac{d}{dt}(\bm{I})_B=0$ with $(\bm{I})_B$ being the inertia tensor of the body defined in the body-fixed frame. Thus, the angular momentum is
\begin{equation}
	\begin{split}
		\sum (\bm{M}^G)_B = (\bm{I}^G)_B \cdot (\bm{\dot{\omega}}^{OB}) + (\bm{\omega}^{OB}) \times \Big[(\bm{I}^G)_B \cdot (\bm{\omega}^{OB})\Big]
		\label{eq:ang_momentum}
	\end{split}
\end{equation}
where $\sum (\bm{M}^G)_B \in \mathbb{R}^{3 \times 1}$ is the sum of the external moments acting on the system around the center of gravity $G$.

The remaining two equations, that are necessary to fully describe the motion of a rigid body in space are the attitude differential equation and the position differential equation. The attitude differential equation describes the relationship between angular rates $p, q, r$ and derivatives of the Euler angles $\dot{\Phi}, \dot{\Theta}, \dot{\Psi}$ leading to
\begin{equation}
	\begin{bmatrix}
		\dot{\Phi}\\
		\dot{\Theta}\\
		\dot{\Psi}
	\end{bmatrix} =
	\begin{bmatrix}
    	1 & \sin\Phi\tan\Theta & \cos\Phi\tan\Theta \\
    	0 & \cos\Phi & -\sin\Phi \\
    	0 & \frac{\sin\Phi}{\cos\Theta} & \frac{\cos\Phi}{\cos\Theta}
    \end{bmatrix}_B
    \begin{bmatrix}
		p\\
		q\\
		r
	\end{bmatrix}_B
	\label{eq:attitude_diff_eq}
\end{equation}
where the angular rates are related to the derivatives of the Euler angles through the \textit{strapdown matrix}.

There are different options to represent the position differential equation. Here, for completeness only, it is written as a simple relationship between the change of the position coordinates in the local \acrshort{NED} frame and the velocity coordinates of the same frame
\begin{equation}
	\begin{bmatrix}
		\dot{x}\\
		\dot{y}\\
		\dot{z}
	\end{bmatrix}_O =
	\begin{bmatrix}
		\dot{V_N}\\
		\dot{V_E}\\
		\dot{V_D}
	\end{bmatrix}_O
	\label{eq:position_diff_eq}
\end{equation}

The Equations (\ref{eq:lin_momentum})-(\ref{eq:position_diff_eq}) presented above can be used to represent a motion of a general aircraft in a three-dimensional space. Note that terms $\sum(\bm{F}^G)_B$ and $\sum (\bm{M}^G)_B$ contain all external forces and moments acting on the rigid body. Considering a general aircraft system, these could be the aerodynamic forces and moments caused by the air flow, propulsion forces and moments, forces caused by the gravitation etc.. The detailed modelling of external forces and moments is presented in chapter \ref{sec:simulink_model}.

\section{Nonlinear Dynamic Inversion} \label{sec:ndi}

In this subsection, the \acrfull{NDI} method is explained. The \acrshort{NDI} approach is based on feedback linearization and is also called \textit{Input-Output Linearization}. Often, such type of controllers is involved in tracking control tasks, where objective is to track some desired trajectory \cite{Slotine}. To derive it, consider the following nonlinear system
\begin{subequations}
	\begin{align}
		&\bm{\dot{x}} = \bm{f}(\bm{x}) + \bm{G}(\bm{x})\bm{u} \label{eq:nonlin_sys1} \\
		&\bm{y} = \bm{h}(\bm{x}) \label{eq:nonlin_sys2}
	\end{align}
	\label{eq:nonlin_sys}
\end{subequations}
\hspace{-5pt}where $\bm{x} \in \mathbb{R}^{n \times 1}$ is the state vector, $\bm{u} \in \mathbb{R}^{m \times 1}$ the input vector, $\bm{y} \in \mathbb{R}^{m \times 1}$ the output vector, $\bm{f}(\bm{x}) \in \mathbb{R}^{n \times 1}$ and $\bm{h}(\bm{x}) \in \mathbb{R}^{m \times 1}$ nonlinear vector fields and $\bm{G} \in \mathbb{R}^{m \times n}$ an input matrix. Note that the system presented in Equations (\ref{eq:nonlin_sys}) is affine in the input, which is not allways fulfilled. Using a state transformation $\bm{z} = \bm{\phi}(\bm{x})$, the affine system from Equation \ref{eq:nonlin_sys1} can be transformed into a \textit{normal} (\textit{canonical}) representation. 

The core idea behind the input-output linearization method is to find a direct relationship between the desired system output and the control input. After the relationship is found it is inverted to generate the control law. To derive this relationship the output $\bm{y}$ is differentiated until the input $\bm{u}$ appears
\begin{equation}
	\begin{split}
		\bm{\dot{y}} &=  \frac{\partial\bm{y}}{\partial t} = \frac{\partial\bm{h}(\bm{x})}{\partial t} \frac{\partial\bm{x}}{\partial t} = \frac{\partial\bm{h}(\bm{x})}{\partial t} \dot{\bm{x}} = \nabla\bm{h}(\bm{x}) [\bm{f}(\bm{x}) + \bm{G}(\bm{x})\bm{u}] \\
		&= \nabla\bm{h}(\bm{x}) \bm{f}(\bm{x}) + \nabla\bm{h}(\bm{x}) \bm{G}(\bm{x})\bm{u} = L_{\bm{f}} \bm{h}(\bm{x}) + L_{\bm{G}} \bm{h}(\bm{x}) \bm{u} 
		\label{eq:dy}
	\end{split}
\end{equation}

In Equation (\ref{eq:dy}) $L_{\bm{f}} \bm{h}(\bm{x})$ is called Lie derivative of $\bm{h}(\bm{x})$ with respect to $\bm{f}(\bm{x})$. The Lie derivative is defined as $L_{\bm{f}} \bm{h}(\bm{x}) = \nabla\bm{h}(\bm{x}) \bm{f}(\bm{x})$ with $\nabla$ being the Nabla operator. Thus, it represents a directional derivative of $\bm{h}(\bm{x})$ along the direction of the vector field $\bm{f}(\bm{x})$. If the term $L_{\bm{G}} \bm{h}(\bm{x})$ is nonzero, the relationship between input and output is
\begin{equation}
	\begin{split}
		\bm{\dot{y}} = L_{\bm{f}} \bm{h}(\bm{x}) + L_{\bm{G}} \bm{h}(\bm{x}) \bm{u} 
		\label{eq:dy_u_rel}
	\end{split}
\end{equation}

Now Equation (\ref{eq:dy_u_rel}) can be used to formulate the control law by solving it for $\bm{u}$ and substituting $\bm{\dot{y}}$ with $\bm{\nu}$
\begin{equation}
	\begin{split}
		\bm{u} = L_{\bm{G}}\bm{h}(\bm{x})^{-1} (\bm{\nu} - L_{\bm{f}}\bm{h}(\bm{x}))
		\label{eq:u_control_law}
	\end{split}
\end{equation}
The variable $\bm{\nu}$ is called an \textit{equivalent input} and represents the desired output of the system. 

In the example provided above the input-output relationsip was found after the first differentiation of the output $\bm{y}$. But if after the first differentiation the term $L_{\bm{G}}\bm{h}(\bm{x})$ is zero, output $\bm{y}$ has to be differentiated until the Lie derivative with respect to $\bm{G}$ is nonzero. The $i$-th derivative of the output is then
\begin{equation}
	\begin{split}
		\frac{\partial^i\bm{y}}{\partial t} = \frac{\partial^i\bm{h}(\bm{x})}{\partial t} = L_{\bm{f}}^i \bm{h}(\bm{x}) + L_{\bm{G}} L_{\bm{f}}^{i-1} \bm{h}(\bm{x}) \bm{u}
		\label{eq:dy_i}
	\end{split}
\end{equation}
with $i$ being the \textit{relative degree} of the system. Using Equation (\ref{eq:dy_i}) to formulate the control law leads to the following expression for the control input $\bm{u}$
\begin{equation}
	\begin{split}
		\bm{u} = L_{\bm{G}} L_{\bm{f}}^{i-1} \bm{h}(\bm{x})^{-1} (\bm{\nu} - L_{\bm{f}}^i\bm{h}(\bm{x})) 
		\label{eq:u_control_law_i}
	\end{split}
\end{equation}
Thus, Equation (\ref{eq:u_control_law_i}) applied to Equation (\ref{eq:dy_i}) yields the simple linear relation
\begin{equation}
	\begin{split}
		\bm{y}^i = \bm{\nu}
		\label{eq:lin_relation}
	\end{split}
\end{equation}
The \acrshort{NDI} method was widely adopted for civil and military aircrafts and has numerous of extensions \cite{Horn}. Nevertheless, it has some drawbacks. The major one is that the control law derived using the \acrshort{NDI} approach is dependent on the full system dynamics model \cite{Sieberling}. The equations of motion of an aircraft usually have a complicated nonlinear character. Thus, describing complex physical phenomena often leads to the inconsistency between the real aircraft and its mathematical representation used in the model. As some of those inconsistencies are inevitable there is a need to build a control law which performance is less dependent on the uncertainties of the model. A method called \acrfull{INDI} can be used to achieve this, it is discussed in the next subsection.

\section{Incremental Nonlinear Dynamic Inversion} \label{sec:indi}

The \acrshort{INDI} is an incremental form of the \acrshort{NDI} for which the lack of the accurate system dynamics model does not critically affect the performance of the control algorithm \cite{Silva}. At first the general form of the \acrshort{INDI} is presented in \ref{subsec:indi_general}, then the inner and outer control loops for the Crazyflie quadrotor are derived in \ref{subsec: indi_inner} and \ref{subsec: indi_outer}.

\subsection{General INDI} \label{subsec:indi_general}

The incremental form of the system can be obtained by taking a Taylor series expansion of the Equation \ref{eq:nonlin_sys1}
\begin{equation}
	\begin{split}
		\bm{\dot{x}} &= \bm{f}(\bm{x}_0) + \bm{G}(\bm{x}_0)\bm{u}_0 \\
		&+ \frac{\partial}{\partial \bm{x}} [\bm{f}(\bm{x}) + \bm{G}(\bm{x})\bm{u}] \bigg|_{\bm{x}=\bm{x}_0,\bm{u}=\bm{u}_0} (\bm{x}-\bm{x}_0) \\
		&+ \frac{\partial}{\partial \bm{u}} [\bm{f}(\bm{x}) + \bm{G}(\bm{x})\bm{u}] \bigg|_{\bm{x}=\bm{x}_0,\bm{u}=\bm{u}_0} (\bm{u}-\bm{u}_0) 
		\label{eq:nonlin_sys_taylor}
	\end{split}
\end{equation}
The first term on the right side of the Equation \ref{eq:nonlin_sys_taylor} is $\bm{\dot{x}_0}$. Also, evaluating the differentiation of the third term leads to 
\begin{equation}
	\begin{split}
		\bm{\dot{x}} &= \bm{\dot{x}}_0 \\ 
		&+ \frac{\partial}{\partial \bm{x}} [\bm{f}(\bm{x}) + \bm{G}(\bm{x})\bm{u}] \bigg|_{\bm{x}=\bm{x}_0,\bm{u}=\bm{u}_0} (\bm{x}-\bm{x}_0) \\
		&+ \bm{G}(\bm{x}_0) (\bm{u}-\bm{u}_0) 
		\label{eq:nonlin_sys_taylor2}
	\end{split}
\end{equation}
The second term of the Equation \ref{eq:nonlin_sys_taylor2} contains partial derivative with respect to the state vector. Considering very small time increments of the controller loop and applying the \textit{principle of time scale separation} the second term vanishes. This is a valid assumption if the dynamics of the actuators is fast compared with the dynamics of the system \cite{Silva}. Thus, the Equation \ref{eq:nonlin_sys_taylor2} is further simplified to 
\begin{equation}
	\begin{split}
		\bm{\dot{x}} = \bm{\dot{x}}_0 + \bm{G}(\bm{x}_0) (\bm{u}-\bm{u}_0) 
		\label{eq:nonlin_sys_indi}
	\end{split}
\end{equation}
By solving Equation \ref{eq:nonlin_sys_indi} for $\bm{u}$ and substituting $\bm{\dot{y}}$ with $\bm{\nu}$, the \acrshort{INDI} control law is obtained
\begin{equation}
	\begin{split}
		\bm{u} = \bm{u}_0 + \bm{G}(\bm{x}_0)^{-1} (\bm{\nu} - \bm{\dot{x}}_0)
		\label{eq:u_control_law_indi}
	\end{split}
\end{equation}
where $\bm{\dot{x}}_0$ is a measurable value from the previous step, $\bm{u_0}$ the control input from the previous step, $\bm{\nu}$ the reference value and $\bm{G}(\bm{x}_0)$ the control effectiveness matrix. With $\bm{\Delta u}=\bm{u}-\bm{u}_0$ the control law from Equation \ref{eq:u_control_law_indi} represents an incremental version of the Equation \ref{eq:u_control_law}. Instead of computing the complete control input command $\bm{u}$, this control law results in computing the increment of the control input $\bm{\Delta u}$ and adding it to the previous value $\bm{u}_0$. As it is less dependent on the model of the system dynamics, it is able to increase the robustness of the system \cite{Sieberling}.

\subsection{Inner INDI loop for Quadrotor} \label{subsec: indi_inner}

In this subsection, using theory from \ref{subsec:indi_general} the inner loop \acrshort{INDI} controller is derived to control angular acceleration of the Crazyflie quadrotor. The derivation of the inner loop, as well as the outer loop controllers is based on the \acrshort{INDI} controller architecture introduced by Smeur \cite{Smeur1}, \cite{Smeur2}. Equation \ref{eq:ang_momentum} from section \ref{sec:eqs_motion} serves as a basis for this derivation. The desired variable to be controlled by the inner loop \acrshort{INDI} is the angular acceleration of the quadrotor in the body-fixed coordinate frame. As in the case of a real quadrotor control problem the value of the thrust can be seen as an output of the dynamic system, it makes sense to incorporate thrust as a control variable into the control law as well. Thus, the angular momentum Equation \ref{eq:ang_momentum} is augmented with the total thrust $T$ of all four rotors \cite{Smeur2}. As only the body-fixed frame is used in the following control law derivation, the subscripts $B$ are not used in this subsection. Also all external forces and moments apply to the center of gravity of the quadrotor, thus the superscripts $G$ are also omitted. 
Solving Equation \ref{eq:ang_momentum} for the angular acceleration results in
\begin{equation}
	\begin{bmatrix}
		\dot{\bm{\omega}}\\
		T
	\end{bmatrix} = 
	\underbrace{
	\begin{bmatrix}
		-\bm{I}^{-1} \big(\bm{\omega} \times \bm{I} \bm{\omega} \big)\\
		0
	\end{bmatrix}}_\text{$\bm{F}(\bm{\omega})$} +
	\underbrace{	
	\begin{bmatrix}
		\bm{I}^{-1} \big( \bm{M}_G + \bm{M}_{A} +  \bm{M}_{P} \big)\\
		T
	\end{bmatrix}}_\text{$\bm{G}(\bm{\omega}, \bm{\Omega}, \bm{\dot{\Omega}})$}
	\label{eq:main_eom_indi_inner}
\end{equation}
where $\bm{\omega}$ and $\dot{\bm{\omega}}$ are angular velocity and acceleration of the body-fixed frame ($B$) with respect to the \acrshort{NED} ($O$) coordinate frame, $\bm{M}_G$ the gravitational moment, $\bm{M}_A$ the aerodynamic moment and $\bm{M}_P$ the propulsion moment. The vector $\bm{\Omega}$ contains angular velocities of all four rotors and serves as an input variable of the system. It is assumed that the gravitational force is applied to the center of gravity of the quadrotor and does not cause any moment around it. Due to the absence of the aerodynamic moment this term is also omitted and can be seen as a disturbance \cite{Smeur1}. The remaining propulsion moment is written as $\bm{M}_P=\bm{M}_C-\bm{M}_{gyro}$ where $\bm{M}_C$ is the control moment generated by the rotors and $\bm{M}_{gyro}$ the moment containing the gyroscopic effect of the rotors. This two moments can be explicitly written as
\begin{equation}
	\bm{M}_C = 	
	\begin{bmatrix}
		-bk_F & bk_F & bk_F & -bk_F \\
		lk_F & lk_F & -lk_F & -lk_F \\
		k_M & -k_M & k_M & -k_M 
	\end{bmatrix} \bm{\Omega}
	\label{eq:m_c}
\end{equation}
\begin{equation}
	\begin{split}
	\bm{M}_{gyro} = 
	\begin{bmatrix}
		0 & 0 & 0 & 0 \\
		0 & 0 & 0 & 0 \\
		I_{rzz} & -I_{rzz} & I_{rzz} & -I_{rzz}
	\end{bmatrix} \bm{\dot{\Omega}}	+
	\begin{bmatrix}
		\bm{\omega}_y & 0 & 0 \\
		0 & \bm{\omega}_x & 0 \\
		0 & 0 & 0 
	\end{bmatrix}	
	\begin{bmatrix}
		I_{rzz} & -I_{rzz} & I_{rzz} & -I_{rzz} \\
		-I_{rzz} & I_{rzz} & -I_{rzz} & I_{rzz} \\
		0 & 0 & 0 & 0 
	\end{bmatrix} \bm{\omega}
	\label{eq:m_gyro}
	\end{split}
\end{equation}
where $I_{rzz}$ is the element of the inertia matrix $I_{r}$ of the rotor, $l$ and $b$ lever arms as denoted in the Figure \ref{fig:frames_leverarms}, $k_F$ and $k_M$ force and moment constants of the rotors.

\begin{figure}[H]
	\centering 
	\begin{tikzpicture}
		\node[inner sep=0pt] (crazyflie_drone) at (0,0) {
		\includegraphics[width=1\textwidth]{figs/frames.pdf}};
	\end{tikzpicture}
	% Figure description is centered aligned
	\captionsetup{justification=centering, singlelinecheck=off, font=bf, belowskip=-0.5cm}
	\caption[Crazyflie 2.1 nano-quadrotor with indicated lever arms of the motors]{Image of the Crazyflie 2.1 quadrotor adopted from \cite{bitcraze} with indicated lever arms of the motors.}
	\label{fig:frames_leverarms}
\end{figure}

As already explained in the previous subsection to derive the incremental control law the Taylor series expansion is perfromed on the Equation \ref{eq:main_eom_indi_inner}
\begin{equation}
	\begin{split}
		\begin{bmatrix}
			\bm{\dot{\omega}}\\
			T
		\end{bmatrix} &= \bm{F}(\bm{\omega}_0) + \bm{G}(\bm{\omega}_0, \bm{\Omega}_0, \bm{\dot{\Omega}}_0) \\
		&+ \frac{\partial}{\partial \bm{\omega}} \big[\bm{F}(\bm{\omega}) + \bm{G}(\bm{\omega}_0, \bm{\Omega}_0, \bm{\dot{\Omega}}_0) \big] \bigg| _{\bm{\omega}=\bm{\omega}_0} (\bm{\omega}-\bm{\omega}_0) \\
		&+ \frac{\partial}{\partial \bm{\Omega}} \big[\bm{G}(\bm{\omega}_0, \bm{\Omega}, \bm{\dot{\Omega}}_0) \big] \bigg| _{\bm{\Omega}=\bm{\Omega}_0} (\bm{\Omega}-\bm{\Omega}_0) \\
		&+ \frac{\partial}{\partial \bm{\dot{\Omega}}} \big[\bm{G}(\bm{\omega}_0, \bm{\Omega}_0, \bm{\dot{\Omega}}) \big] \bigg| _{\bm{\dot{\Omega}}=\bm{\dot{\Omega}}_0} (\bm{\dot{\Omega}}-\bm{\dot{\Omega}}_0)
		\label{eq:main_eom_indi_inner_taylor}
	\end{split}
\end{equation}
By applying differentiation and rewriting some of the terms, following equation is obtained
\begin{equation}
	\begin{bmatrix}
		\bm{\dot{\omega}}\\
		T
	\end{bmatrix} = 
	\begin{bmatrix}
		\bm{\dot{\omega}}_0\\
		T_0
	\end{bmatrix} + \bm{G}_1(\bm{\Omega} - \bm{\Omega}_0) + T_S \bm{G}_2(\bm{\dot{\Omega}} - \bm{\dot{\Omega}}_0)
	\label{eq:main_eom_indi_inner_lin}
\end{equation}
The first term of the Equation \ref{eq:main_eom_indi_inner_lin} is the angular acceleration based on the current angular rates $\bm{\omega}_0$ and inputs $\bm{\Omega}_0$ and can be denoted as $\bm{\dot{\omega}}_0$. $T_0$ is the current thrust value. The last term on the right side of the Equation \ref{eq:main_eom_indi_inner_lin} is scaled with the sample time $T_s$ which is introduced only to simplify further mathematical transformations. The expressions of the control moment $\bm{M}_C$ and the gyroscopic moment of the rotors $\bm{M}_{gyro}$ have been summarized to control effectiveness matrices $\bm{G}_1$ and $\bm{G}_2$
\begin{equation}
	\bm{G}_1 = 
	\begin{bmatrix}
		\bm{I}^{-1}\begin{bmatrix}
			-bk_F & bk_F & bk_F & -bk_F \\
			lk_F & lk_F & -lk_F & -lk_F \\
			k_M & -k_M & k_M & -k_M 
		\end{bmatrix} - 	
		\begin{bmatrix}
			\bm{\omega}_y & 0 & 0 \\
			0 & \bm{\omega}_x & 0 \\
			0 & 0 & 0 
		\end{bmatrix}	
		\begin{bmatrix}
			I_{rzz} & -I_{rzz} & I_{rzz} & -I_{rzz} \\
			-I_{rzz} & I_{rzz} & -I_{rzz} & I_{rzz} \\
			0 & 0 & 0 & 0 
		\end{bmatrix}\\
			k_F \cdot \bm{1}_{1\times4}
	\end{bmatrix}
	\label{eq:G1}
\end{equation}
\begin{equation}
	\bm{G}_2 = 
	\begin{bmatrix}
		-\bm{I}^{-1} T_S^{-1}\begin{bmatrix}
			0 & 0 & 0 & 0 \\
			0 & 0 & 0 & 0 \\
			I_{rzz} & -I_{rzz} & I_{rzz} & -I_{rzz}
		\end{bmatrix} \\
		\bm{0}_{1\times4}
	\end{bmatrix}
	\label{eq:G2}
\end{equation}
with the terms $\bm{1}_{1\times4}$ and $\bm{0}_{1\times4}$ being $1\times4$ vectors of ones and zeros, respectively.

To prepare the linearized dynamic Equation \ref{eq:main_eom_indi_inner_lin} for discrete implementation on a computing system, the discrete approximation ($z$ domain) of the derivative is used: $\bm{\dot{\Omega}} = (\bm{\Omega} - \bm{\Omega}z^{-1})T_S^{-1}$. Furthermore, angular acceleration $\bm{\dot{\omega}}_0$ which is obtained from the differentiated gyroscope measurements is usually noisy. It has been shown, that using a second order filtering can help to reduce measurement nose. At the same time such a filter introduces time delay, which has to be considered in the derivation, as it is important to have a unique delay for all variables which are used in the Taylor expansion \cite{Smeur1}. Thus, the same second order filter is applied to all variables with a subscript $0$. By applying filtering (subscript $f$) and finite differences method to the Equation \ref{eq:main_eom_indi_inner_lin}, its discrete version results in
\begin{equation}
	\begin{bmatrix}
		\bm{\dot{\omega}}\\
		T
	\end{bmatrix} = 
	\begin{bmatrix}
		\bm{\dot{\omega}}_f\\
		T_f
	\end{bmatrix} + (\bm{G}_1+\bm{G}_2)(\bm{\Omega} - \bm{\Omega}_f) - \bm{G}_2z^{-1}(\bm{\Omega} - \bm{\Omega}_f)
	\label{eq:main_eom_indi_inner_lin_final}
\end{equation}
By solving Equation \ref{eq:main_eom_indi_inner_lin_final} for $\bm{\dot{\Omega}}$ and substituting $\bm{\dot{\omega}}$ with the equivalent input $\bm{\nu}_{ang}$, the control law of the inner loop is obtained
\begin{equation}
	\begin{split}
		\bm{\dot{\Omega}}_c = \bm{\dot{\Omega}} = \bm{\dot{\Omega}}_f + (\bm{G}_1+\bm{G}_2)^{-1} \Bigg(
		\begin{bmatrix}
			\bm{\nu}_{ang} - \bm{\dot{\omega}}_f\\
			\tilde{T}
		\end{bmatrix} + \bm{G}_2z^{-1}(\bm{\Omega} - \bm{\Omega}_f) \Bigg)
		\label{eq:indi_inner_control_law}	
	\end{split}
\end{equation}
where $\bm{\dot{\Omega}}$ is a vector of commanded rotational rates for every rotor and $\tilde{T} = T - T_f$ being the thrust increment which is provided by the outer loop \acrshort{INDI}. The block diagram of the inner loop \acrshort{INDI} controller is presented in Figure (TODO:Figure). In the figure the derived controller which controls the angular acceleration is augmented with two simple controllers which control the angular velocity $\bm{\omega}$ and the attitude $\bm{\eta}$ of the quadrotor. Each controller consists of a single gain. Thus, the reference values of the \acrshort{INDI} part of the inner loop are provided by these two controllers. 

\subsection{Outer INDI loop for Quadrotor} \label{subsec: indi_outer}

In this subsection the derivation of the outer loop \acrshort{INDI} controller is introduced. The outer loop \acrshort{INDI} controls translational acceleration of the quadrotor. Linear momentum Equation \ref{eq:lin_momentum} serves as a basis for this derivation. For the simplicity of the further transformations the derivation is performed in the \acrshort{NED} frame (subscript $O$). Thus, the Equation \ref{eq:lin_momentum} becomes
\begin{equation}
	\begin{split}
		\sum (\bm{F}^G)_O = m \cdot (\bm{\dot{v}}^G)_{O}
		\label{eq:lin_momentum_indi_outer}
	\end{split}
\end{equation}
As it is assumed that all forces apply to the center of gravity of the quadrotor, the superscript $G$ is omitted in the future. Thus, solving Equation \ref{eq:lin_momentum_indi_outer} for $\bm{\dot{v}}$ and assuming that the sum of all forces acting on the quadrotor consists only of gravitational, propulsive and aerodynamic forces, following equation is obtained
\begin{equation}
	\begin{split}
		\bm{\dot{v}} = m^{-1} ((\bm{F}_G)_O + (\bm{F}_P)_O+ (\bm{F}_A)_O)
		\label{eq:lin_momentum_indi_outer_simple}
	\end{split}
\end{equation}
The aerodynamic force $(\bm{F}_A)_O$ is modelled as an unknown function of velocity $\bm{v}$ and wind vector $\bm{\chi}$. For the gravitational force $(\bm{F}_G)_O$ the simplest one-dimensional gravity model is assumed.
\begin{equation}
	\begin{split}
		(\bm{F}_A)_O = \bm{f}(\bm{v}, \bm{\chi})
		\label{eq:Fa}
	\end{split}
\end{equation}
\begin{equation}
	\begin{split}
		(\bm{F}_G)_O = m 
		\begin{bmatrix}
			0\\
			0\\
			g
		\end{bmatrix}_O
		\label{eq:Fg}
	\end{split}
\end{equation}
The propulsive force $(\bm{F}_P)_O$ (produced by the rotors) has only one nonzero component if defined in the body-fixed frame. To transform it to the \acrshort{NED} frame the rotation matrix $\bm{M}_{OB}$ is used. This results in 
\begin{equation}
	\begin{split}
		(\bm{F}_P)_O = \bm{M}_{OB}
		\begin{bmatrix}
			0\\
			0\\
			-T
		\end{bmatrix}_B = 
		\begin{bmatrix}
			* & * & c\Psi s\Theta c\Phi + s\Psi s\Phi\\
			* & * & s\Psi s\Theta c\Phi + c\Psi s\Phi\\
			* & * & c\Theta c\Phi
		\end{bmatrix}
		\begin{bmatrix}
			0\\
			0\\
			-T
		\end{bmatrix}_B = 
		\begin{bmatrix}
			(c\Psi s\Theta c\Phi + s\Psi s\Phi)(-T) \\
			(s\Psi s\Theta c\Phi + c\Psi s\Phi)(-T) \\
			(c\Theta c\Phi)(-T)
		\end{bmatrix}_O
		\label{eq:Fp}
	\end{split}
\end{equation}
To obtain the incremental form of the linear momentum equation, the same strategy as in the case of the inner loop derivation is applied: the Equation \ref{eq:lin_momentum_indi_outer_simple} is linearized using Taylor series expansion
\begin{equation}
	\begin{split}
		\bm{\dot{v}} &= m^{-1} [\bm{F}_G + \bm{F}_P(\Phi_0, \Theta_0, \Psi_0, T_0)+ \bm{F}_A(\bm{v}_0, \bm{\chi}_0)]  \\		
		&+ \frac{\partial}{\partial \bm{v}} \big[ \bm{F}_A(\bm{v}, \bm{\chi}_0) \big] \bigg| _{\bm{v}=\bm{v}_0} (\bm{v}-\bm{v}_0) \\
		&+ \frac{\partial}{\partial \bm{\chi}} \big[ \bm{F}_A(\bm{v}_0, \bm{\chi}) \big] \bigg| _{\bm{\chi}=\bm{\chi}_0} (\bm{\chi}-\bm{\chi}_0) \\
		&+ \frac{\partial}{\partial \Phi} \big[ \bm{F}_P(\Phi, \Theta_0, \Psi_0, T_0) \big] \bigg| _{\Phi=\Phi_0} (\Phi-\Phi_0) \\
		&+ \frac{\partial}{\partial \Theta} \big[ \bm{F}_P(\Phi_0, \Theta, \Psi_0, T_0) \big] \bigg| _{\Theta=\Theta_0} (\Theta-\Theta_0) \\
		&+ \frac{\partial}{\partial \Psi} \big[ \bm{F}_P(\Phi_0, \Theta_0, \Psi, T_0) \big] \bigg| _{\Psi=\Psi_0} (\Psi-\Psi_0) \\
		&+ \frac{\partial}{\partial T} \big[ \bm{F}_P(\Phi_0, \Theta_0, \Psi_0, T) \big] \bigg| _{T=T_0} (T-T_0) \\
		\label{eq:main_eom_indi_outer_taylor}
	\end{split}
\end{equation}
The first term on the right side of the Equation \ref{eq:main_eom_indi_outer_taylor} is the acceleration at previous time step $\bm{\dot{v}}_0$. It can be obtained from the onboard accelerometer and then tranformed to the \acrshort{NED} frame. The next two terms on the right are assumed to be zero since it is very complicated to develop an accurate model of the aerodynamic force as the nature of the wind influence can vary and remains unknown. It is also assumed that the change of the yaw angle $\Psi$ is small and can be neglected as well. By applying considered assumptions Equation \ref{eq:main_eom_indi_outer_taylor} is written as 
\begin{equation}
	\begin{split}
		\bm{\dot{v}} = \bm{\dot{v}}_0 +  m^{-1} \bm{G}(\Phi_0, \Theta_0, \Psi_0, T_0) (\bm{u} - \bm{u}_0)
		\label{eq:lin_momentum_indi_outer_lin}
	\end{split}
\end{equation}
where
\begin{equation}
	\begin{split}
		\bm{G}(\Phi_0, \Theta_0, \Psi_0, T_0) = 
		\underbrace{
		\begin{bmatrix}
			(-c\Psi_0 s\Theta_0 s\Phi_0 + s\Psi_0 c\Phi_0)T_0  &%
			c\Psi_0 c\Theta_0 c\Phi_0 T_0 &%
			c\Psi_0 s\Theta_0 c\Phi_0 + s\Psi_0 s\Phi_0 \\			
			(-s\Psi_0 s\Theta_0 s\Phi_0 - c\Psi_0 c\Phi_0)T_0 &%
			s\Psi_0 c\Theta_0 c\Phi_0 T_0 &% 
			s\Psi_0 s\Theta_0 c\Phi_0 - c\Psi_0 s\Phi_0\\
			-c\Theta_0 s\Phi_0T_0 &%
			-s\Theta_0 c\Phi_0T_0 &%
			c\Theta_0 c\Phi_0
		\end{bmatrix}}_\text{$\bm{G}(\Phi_0, \Theta_0, \Psi_0, T_0)$}
		\label{eq:G}
	\end{split}
\end{equation}
and
\begin{equation}
	\begin{split}
		\bm{u} = 
		\begin{bmatrix}
			\Phi\\
			\Theta\\
			T
		\end{bmatrix}
		\label{eq:u_outer}
	\end{split}
\end{equation}
$\bm{G}(\Phi_0, \Theta_0, \Psi_0, T_0)$ is the control effectiveness matrix which is computed based on the attitude and thrust values from the previous step. $\bm{u}$ is a vector of control variables which are passed to the inner loop.

The linear acceleration $\bm{\dot{v}}_0$ which is obtained from accelerometer measurements is usually noisy. Applying the same filter as in the case of the inner loop derivation, the Equation \ref{eq:lin_momentum_indi_outer_lin} is given by 
\begin{equation}
	\begin{split}
		\bm{\dot{v}} = \bm{\dot{v}}_f +  m^{-1} \bm{G}(\Phi_0, \Theta_0, \Psi_0, T_0) (\bm{u} - \bm{u}_f)
		\label{eq:lin_momentum_indi_outer_lin_final}
	\end{split}
\end{equation}
Solving Equation \ref{eq:lin_momentum_indi_outer_lin_final} for $\bm{u}$ and substituting $\bm{\dot{v}}$ with the equivalent control variable $\bm{\nu}_{lin}$, the control law of the outer loop is obtained
\begin{equation}
	\begin{split}
		\bm{u_c} = \bm{u} = \bm{u}_f +  m \bm{G}^{-1}(\Phi_0, \Theta_0, \Psi_0, T_0) (\bm{\nu}_{lin} - \bm{\dot{v}}_f)
		\label{eq:indi_outer_control_law}
	\end{split}
\end{equation}
Figure (TODO:show Figure) shows the block diagram of the outer loop \acrshort{INDI}. Using the same approach as with the inner loop, the outer loop which controls linear acceleration is augmented with two closed loops consisting of two gains. These gains represent two controllers which control linear velocity and position. 

TODOs: \\
- INDI inner loop (order)\\
- 2.3.2 Figure with Crazyflie lever arms and motor numbers \\
- 2.3.2 Figure with block diagram of the inner loop controller\\
- 2.3.3 Figure with block diagram of the outer loop controller

\chapter{Implementation} \label{cha:implementation}

This chapter describes different aspects of the practical realization of the control architecture introduced in the previous chapter. Section \ref{sec:research_quadrotor} gives an overview about the quadrotor system on which the \acrshort{INDI} outer loop controller was implemented and tested. In section \ref{sec:simulink_model} the simulation model of the Crazyflie quadrotor with implemented \acrshort{INDI} inner and outer loops is presented. Finally, in section \ref{sec:implementation_on_hardware} the process of the controller implementation on the hardware system is described. Here, additionally to the complete outer loop implementation, it is described how the already existing inner loop was modified to achieve a better performance. 

% Manually add Chapter names to the header (per default not included by Latex)
\thispagestyle{fancy}
\chaptermark{Implementation}

\section{Research Quadrotor} \label{sec:research_quadrotor}

The Crazyflie 2.1 quadrotor developed by Swedish company Bitcraze belongs to the nano-quadrotor class (see Figure \ref{fig:crazyflie_drone}). Its total weight ranges from 27g to 42g, which makes possible to extend the default hardware with a pair of additional extension decks of various functionality. As it is an open source project, the main firmware, development tools and other software packages are freely available. This is one of the reasons why this small quadrotor gained a lot of popularity in different academia fields, being used for tasks reaching from the classical control theory all the way down to the obstacle avoidance problems. 

The Crazyflie's main application \acrfull{MCU} is STM32F405. It runs the firmware with frequency of 1000 Hz and communicates with a large number of peripheral devices such as Bluetooth and long range radio, power management system \acrshort{MCU}, \acrfull{IMU} etc.. The \acrfull{IMU} consists of the 3-axis accelerometer, the 3-axis gyroscope and the high precision pressure sensor. 

The quadrotor used in this project was extended with a flow deck sensor. The flow deck itself is equipped with two sensors: the optical flow sensor which measures the visual motion and the \acrfull{ToF} sensor which measures the range to the next object along the body-fixed z-axis of the quadcopter. By employing these sensors, the flow deck can provide estimates of the current position and linear velocity.

\begin{figure}[H]
	\centering 
	\begin{tikzpicture}
		\node[inner sep=0pt] (crazyflie_drone) at (0,0) {
		\includegraphics[width=0.4\textwidth]{figs/crazyflie_drone.jpg}};
	\end{tikzpicture}
	% Figure description is centered aligned
	\captionsetup{justification=centering, singlelinecheck=off, font=bf, belowskip=-0.5cm}
	\caption[Crazyflie 2.1 nano-quadrotor]{Crazyflie 2.1 nano-quadrotor \cite{bitcraze}}
	\label{fig:crazyflie_drone}
\end{figure}

\section{Simulink Model} \label{sec:simulink_model}

Before implementing the outer loop of the \acrshort{INDI} controller derived in \ref{subsec: indi_outer} on the real Crazyflie quadrotor it was decided firstly to build a simulation model using Matlab/Simulink software. Such a simulation has a number of advantages compared to the direct implementation on the hardware. The main one is, it allows one to conveniently test different model components (e.g. filtering algorithms, actuator dynamic models etc.) on their dedicated functions, facilitating the debugging process. It also can ease the process of parameter tuning of the controller. 

In the framework of this thesis the Simulink model was also applied to better understand the controller structure and its characteristics. Based on the simulation it was determined which of the control effectiveness terms have a greater influence on the system response and which ones can be neglected. Additionally the simulation model provided a good overview of potential computational problems and mistakes which could lead to drastically growing values and unstable behaviour of the system. To solve some of them, saturation blocks have been used  in different places.

\subsection{Quadrotor Dynamics} \label{subsec:quadrotor_dynamics}

The core of the Crazyflie simulation is the model of the quadrotor dynamics. It is based on dynamic equations of motion presented in Chapter \ref{sec:eqs_motion}. Similarly to the section \ref{subsec: indi_outer}, it was assumed that the sum of the forces in the linear momentum equation is composed of three components, namely gravitational, propulsive and aerodynamic. The gravity and the propulsive force have not been changed, so the Equations \ref{eq:Fg} and \ref{eq:Fp} were also adopted for the simulation model. The thrust value $T$ was modelled as a linear function of rotational rates of propellers 
\begin{equation}
	\begin{split}
		T = - k_F (\Omega_1 + \Omega_2 + \Omega_3 + \Omega_4)
		\label{eq:thrust}
	\end{split}
\end{equation}
where $k_F$ is the force constant of the rotors. In subsection \ref{subsec:parameter_estimation} it is shown how to estimate the thrust coefficient directly from the flight data. However, for the simulation model $k_F$ was computed based on the assumption that the influence of the force constant of the rotors $k_F$ (considering the lever arm) is greater than the influence of the moment constant of the rotors $k_M$, and that the thrust to weight ratio expression $\frac{T}{W} \approx 2$.

To account for aerodynamic effects such as drag the aerodynamic force was approximated with following expression 
\begin{equation}
	\begin{split}
		(\bm{F}_A)_B =  
		\begin{bmatrix}
			-k_x (u_K)_B\\
			-k_y (v_K)_B\\
			-k_z (w_K)_B
		\end{bmatrix}_B
		\label{eq:Fa_simulink}
	\end{split}
\end{equation}
where $(u_K)_B$, $(v_K)_B$ and $(w_K)_B$ are kinematic velocities of the quadrotor along the axes $x_K$, $y_K$ and $z_K$ respectively, given in the body-fixed coordinate frame. $k_x$, $k_y$ and $k_z$ are experimentally estimated drag coefficients adopted from \cite{foerster}.

The angular momentum equation used in the simulation model has the same form as defined in Equation \ref{eq:ang_momentum}. The numerical values of the inertia tensor $(\bm{I}^G)_B$ were estimated using pendulum method in \cite{foerster}. The total external moment acting on the Crazyflie consists only of the propulsive moment $\bm{M}_P$ as it is defined in \ref{subsec: indi_inner}. It consist of the control moment generated by the rotors $\bm{M}_C$  and moment containing the gyroscopic effect of the rotors $\bm{M}_{gyro}$ (see Equations \ref{eq:m_c} and \ref{eq:m_gyro}). The moment constant of the rotor was computed to fulfil the requirements on $k_F$ and $k_M$ which were described above. To obtain the numerical estimate of the propeller inertia matrix $I_{rzz}$, the propeller was approximated with a bar element. This has been shown to be a valid assumption since the influence of the propeller inertia (due to its negligible weight) is very small.

TODOs:\\
- 3.2.1 reference to the table with all used constants in the appendix

\subsection{Actuator Dynamics} \label{subsec:actuator_dynamics}

The inner loop \acrshort{INDI} controller relies on the measurement of the rotor angular rates. If the actuator feedback is not available, it is possible to use its model instead. The model of the actuator dynamics was assumed to be a first order lag filter with following transfer function in continuous Laplace domain
\begin{equation}
	\begin{split}
		A(s) = \frac{k}{1+T_S}
		\label{eq:act_dynamics_c}
	\end{split}
\end{equation}
where $k$ is the gain and $T$ the time constant. Section \ref{sec:implementation_on_hardware} shortly describes the process of using flight data to estimate the time constant $T$. In the Simulink model it was assumed that the actuator feedback is available, hovewer for the implementation on hardware it is necessary to have a discrete version of the transfer function in Equation \ref{eq:act_dynamics_c}. Transforming the transfer function in Equation \ref{eq:act_dynamics_c} from continuous domain into discrete by applying \textit{Zero-Order Hold method} results in
\begin{equation}
	\begin{split}
		A(z) = \frac{k\alpha}{z+(\alpha-1)}
		\label{eq:act_dynamics_d}
	\end{split}
\end{equation}
with $\alpha = 1-e^{\frac{-T_S}{T}}$ and sample time $T_S$. The default frequency of the Crazyflie controller update is 500 Hz. Thus, sampling time $T_S$ was set to the reciprocal value of the default frequency. 

\subsection{Filtering} \label{subsec:filtering}

As it was shown in Chapter \ref{cha:theoretical_background} (TODO also see Figures ... and ...), the inner and outer loops employ several second order filters. The main task of two of them is to reduce  noise in the measurements of the angular and linear acceleration. Other filters are applied to remaining feedback quantities. Using the same parameters for every filter, this technique accounts for the time delay which is introduced by the filters and ensures that all quantities that were fed back are from the same point in time. The gneral transfer function of a second order filter in continuous domain is defined by the following formula
\begin{equation}
	\begin{split}
		F(s) = \frac{\omega_n^2}{s^2+2\zeta\omega_ns+\omega_n^2}
		\label{eq:filter_c}
	\end{split}
\end{equation}
with a relative damping coefficient $\zeta$ and an eigenfrequency $\omega_n$. To implement transfer function from Equation \ref{eq:filter_c} on a digital computer it must be transformed to the discrete domain. To perform this transformation \textit{bilinear transformation} (also called \textit{Tustin-Transformation}) was used. Using bilinear transformation, the mapping relation between $s$ and $z$ results in
\begin{equation}
	\begin{split}
		s \ \widehat{=} \ \frac{2}{T_S} \frac{1-z^{-1}}{1+z^{-1}}
		\label{eq:filter_c}
	\end{split}
\end{equation}
Thus, the discrete version of the Equation \ref{eq:filter_c} is 
\begin{equation}
	\begin{split}
		F(z) = \frac{n_0 + n_1z^{-1} + n_2z^{-2}}{d_0 + d_1z^{-1} + d_2z^{-2}}
		\label{eq:filter_d}
	\end{split}
\end{equation}
where 
\begin{subequations}
	\begin{align}
		n_0 &= \omega_n^2 \\
		n_1 &= 2\omega_n^2 \\
		n_2 &= \omega_n^2 \\
		d_0 &= \bigg[(\frac{2}{T_S})^2 + \frac{4\zeta \omega_n}{T_S} + \omega_n^2\bigg] \\
		d_1 &= \bigg[\frac{8}{T_S^2} + 2\omega_n^2\bigg] \\
		d_2 &= \bigg[(\frac{2}{T_S})^2 - \frac{4\zeta \omega_n}{T_S} + \omega_n^2\bigg]
	\end{align}
	\label{eq:filter_parameters}
\end{subequations}

\subsection{Simulation Results and Analysis} \label{subsec:simulation_results}

- Inner loop subsubsection (Tell that it is descrete implementation and then analysis) \\
- Outer loop subsubsection (- // -) \\
- Appendix: simulink model images, table with all constants \\
- Plot differences with or without some terms (G1, G2\_yaw) \\
- Responces for different step inputs
- pz-Plot

TODOs:\\
- 3.2.2 'Actuator Dynamics' Fig4 from Adaptive... Smeur paper
- 3.2.3 'Filtering' Reference figures of the inner and outer block diagrams
- 3.2.3 'Filtering' Block diagram of the filter 
\section{Implementation on the Hardware} \label{sec:implementation_on_hardware}

\subsection{Parameter Estimation} \label{subsec:parameter_estimation}

- Actuator Dynamics (Estimation of the time constant), plot response with estimated constant\\
- Thrust dynamics estimation\\
- Estimation of the control effectiveness parameters G1, G2 for inner INDI (describe performed flight to log data), plot the curve with contribution of G1, G2 to the fitting\\
- PD gain tuning for inner INDI (using pd\_inner\_cs(), show plots with different D-gains (e.g 25, 10 and 3) to see different damping behaviour), Before explaining gain tuning present all relevant transfer functions of the closed and open loops\\
- Outer loop (show new diagram of the controller)

\subsection{Structure of the Code} \label{subsec:structure_ofthe_code}

- Moore-Penrose instead of the inverse 

\subsection{Testing with contact Forces and Moments?} \label{subsec:testing_with_fandm}

\chapter{Results} \label{cha:results}

% Manually add Chapter names to the header (per default not included by Latex)
\thispagestyle{fancy}
\chaptermark{Results}

- To make the controller work a minimal knowledge of the system dynymics is need. Nevertherless G1, G2 still have to estimated accuratly because this 2 parameters do have inpact on the controller performance. \\
- Plain step responses of the outer loop \\
- Step responses with disturbance of the outer loop


\chapter{Discussion} \label{cha:discussion}

% Manually add Chapter names to the header (per default not included by Latex)
\thispagestyle{fancy}
\chaptermark{Discussion}

\newpage

% Activate and reset roman page numbering
\pagenumbering{roman}
\setcounter{page}{1}

% Add references
\renewcommand\bibname{References}		% Rename Bibliography to References
\bibliographystyle{unsrt}
\bibliography{Lit/Literature}

% Manually add Chapter names to the header (per default not included by Latex)
\thispagestyle{fancy_beginning}
\renewcommand{\chaptermark}[1]{\markboth{#1}{}}
\chaptermark{References}

\chapter*{Appendix} \label{sec:appendix}
% Add Appendix chapter manually to the Table of Contents because Latex doesn't automatically include nonnumerated (...*) chapters/sections to the TOC
\addcontentsline{toc}{chapter}{Appendix}

% Manually add Chapter names to the header (per default not included by Latex)
\thispagestyle{fancy_beginning}
\renewcommand{\chaptermark}[1]{\markboth{#1}{}}
\chaptermark{Appendix}

% End of the document
\end{document}


%\begin{equation}
%	\begin{bmatrix}
%		\dot{\bm{\omega}}\\
%		T
%	\end{bmatrix} = \bm{F}(\bm{\omega}) + \bm{G}(\bm{\omega}, \bm{\Omega}, \bm{\dot{\Omega}})
%	\label{eq:TODO}
%\end{equation}