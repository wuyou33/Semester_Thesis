% This is one possible realization of the official FSD Microsoft Word template. The title page has to be filled in in Microsoft Word environment and saved as a PDF file. Note, that in the header and footer the name of the author and the thesis topic have to be provided. 
%
% Compile Template in following order:
%
% pdflatex Template
% makeindex Template.nlo -s nomencl.ist -o Template.nls
% makeindex Template.acn -s Template.ist -t Template.alg -o Template.acr
% makeindex Template.glo -s Template.ist -t Template.glg -o Template.gls
% pdflatex Template
% pdflatex Template
%
%
%#################################################################################
%#################################################################################
%#################################################################################

% Preamble
\documentclass[11pt, a4paper, twoside]{report}

% Define page layout
\usepackage[
 left=2.5cm,
 right=2.5cm,
 top=2.5cm,
 bottom=3cm]
{geometry}

% Package for defining the default font size
% We use Helvetica as an alternative to Arial
\usepackage{helvet}
\renewcommand{\familydefault}{\sfdefault}
%alternatively one can use Arial package and compile with XeLaTEX
%\usepackage{fontspec}
%\setmainfont{Arial}

% Set line spacing to 1.15pt
\renewcommand{\baselinestretch}{1.15}

% Set spacing between paragraphs to be 10pt
\setlength{\parskip}{10pt}

% Set paragraph intendation (here 0pt, no intendation)
\setlength{\parindent}{0pt}

% Package for estimating the last page
\usepackage{pageslts}

% Package and settings for adding header/footer
\usepackage{fancyhdr}

% Set style for pages with arabic numbering 
% This block prints only the chapter number and the chapter name, not the word "Chapter"
%\renewcommand{\chaptermark}[1]{\markboth{\MakeUppercase{\ \thechapter. \ #1}}{}}
% Clear current page style 
\fancyhf{}
% Head (11 pt font size and 12 pt line spacing)
\fancyhf[HRE,HLO]{\includegraphics[width=2.5cm]{figs/fsd}}
\fancyhf[HRO,HLE]{\fontsize{11}{12}\selectfont \leftmark}
% Foot (11 pt font size and 12 pt line spacing)
\fancyhf[FRE,FLO]{\fontsize{11}{12}\selectfont Title\\First Name Last Name}
\fancyhf[FLE,FRO]{\fontsize{11}{12}\selectfont Page \thepage \thinspace / \lastpageref{LastPages}}
% Line settings
\renewcommand{\headrulewidth}{0.1pt}	%upper line
\renewcommand{\footrulewidth}{0.1pt}	%lower line

% Set style for pages with roman numbering 
\fancypagestyle{fancy_beginning}{%
%\pagestyle{fancy}
% Clear current page style 
\fancyhf{}%
% Head
\fancyhf[HRE,HLO]{\includegraphics[width=2.5cm]{figs/fsd}}%
\fancyhf[HRO,HLE]{\fontsize{11}{12}\selectfont \leftmark}%
% Foot
\fancyhf[FRE,FLO]{\fontsize{11}{12}\selectfont Title\\First Name Last Name}%
\fancyhf[FLE,FRO]{\fontsize{11}{12}\selectfont Page \thepage}%
% Line settings
\renewcommand{\headrulewidth}{0.1pt}%	%upper line
\renewcommand{\footrulewidth}{0.1pt}%	%lower line
}

% Package for inserting titel page
\usepackage{pdfpages}

% Package for changing distance between sections
\usepackage{titlesec} 

% Package for setting up the caption options of figures
\usepackage{subcaption}

% Package prevents placing floats before a section (for not mooving figures)
\usepackage{float}

% TODO
% Package for using acronyms 
%\usepackage[printonlyused]{acronym}
%\renewcommand\acsfont{\textnormal}	% Set non-bold font of the acronym units

% This patch removes the insertion of \addvspace to both the LoF and the LoT; the cause of the additional gap between entries on a per-chapter basis
\usepackage{etoolbox}% http://ctan.org/pkg/etoolbox
\makeatletter
% \patchcmd{<cmd>}{<search>}{<replace>}{<succes>}{<failure>}
\patchcmd{\@chapter}{\addtocontents{lof}{\protect\addvspace{10\p@}}}{}{}{}% LoF
\patchcmd{\@chapter}{\addtocontents{lot}{\protect\addvspace{10\p@}}}{}{}{}% LoT
\makeatother

% Set chapter font size and spacing
\newcommand{\chapfnt}{\fontsize{14}{19}}
\titleformat{\chapter}
{\normalfont\chapfnt\bfseries}{\thechapter}{1em}{\chapfnt} 	% set font size
\titlespacing{\chapter}{0cm}{-7mm}{0cm}					% set spacing

% Set section font size (13pt) and spacing
\newcommand{\secfnt}{\fontsize{13}{19}}
\titleformat{\section}
{\normalfont\secfnt\bfseries}{\thesection}{1em}{}
\titlespacing*{\section}{0cm}{0.3cm}{0cm}					% set spacing

% Package for edditing TOC, LOF and LOT
\usepackage[titles]{tocloft}

% Package for setting itemization/description parameters like indentation of items
\usepackage{enumitem}

% Package for acronyms
\usepackage[acronym, automake, nopostdot, nogroupskip]{glossaries}
\setlength{\LTleft}{-0.15cm}			% left justified
\renewcommand*{\arraystretch}{1.4}		% vertical distance between entries, default is 1
% List
\newacronym{ADF}{ADF}{Automatic Direction Finder}
\newacronym{ADI}{ADI}{Automatic Direction Indicator}
% Make gls
\renewcommand*\entryname{Acronym}
\makeglossaries

% Package for setting up the list of symbols
%https://tex.stackexchange.com/questions/112884/how-to-achieve-nomenclature-entries-like-symbol-description-dimension-and-uni
\usepackage{nomencl,etoolbox,ragged2e,siunitx}
%
\newcommand{\DimensUnits}[2]{\makebox[1em]{#1}\makebox[4em]{#2\hfill}\ignorespaces}
\newcommand{\insertnomheaders}{\item[\bfseries Symbol]\DimensUnits{}{\textbf{Unit}}\textbf{Description}}
%
\renewcommand\nomgroup[1]{%
  \item[\large\bfseries
  \ifstrequal{#1}{A}{Acronyms}{%
  \ifstrequal{#1}{B}{Latin Letters}{%
  \ifstrequal{#1}{G}{Greek Letters}{%
  \ifstrequal{#1}{S}{Indices}{% 
  \ifstrequal{#1}{U}{Subscripts}{%   
  \ifstrequal{#1}{X}{Other Symbols}{}}}}}}]
  \insertnomheaders
  }
%
\renewcommand*{\nompreamble}{\markboth{\nomname}{\nomname}}
% parbox - Description
\newcommand{\nomdescr}[1]{\parbox[t]{10cm}{\RaggedRight #1}}
\newcommand{\nomwithdim}[5]{\nomenclature[#1]{#2}{\DimensUnits{}{#5}\nomdescr{#3}}}
%
\makenomenclature


%#################################################################################%#################################################################################%#################################################################################

\begin{document}

% Activate roman page numbering
\pagenumbering{roman}

% Insert title page
\includepdf[pages={1}]{figs/Title_page.pdf}

%Insert empty page after title page
\newpage\null\thispagestyle{empty}\newpage


% For some reason fisrt chapter has a different vert. spacing from the top, so we add additional 0.8cm, starred command makes sure it will not be suppressed at the beginning or at the end of the page
\vspace*{-0.8cm}
\section*{Statutory Declaration}
I, , declare on oath towards the Institute of Flight System Dynamics of Technische Universität München, that I have prepared the present BA/SA/MA independently and with the aid of nothing but the resources listed in the bibliography.
\\
\\
This thesis has neither as-is nor similarly been submitted to any other university.
\bigskip
\\
\\
Garching, 

% Manually add Chapter names to the header (per default not included by Latex)
\thispagestyle{fancy_beginning}
\renewcommand{\chaptermark}[1]{\markboth{#1}{}}
\chaptermark{Statutory Declaration}

%Insert empty page after statutory declaration section
\newpage\null\thispagestyle{empty}\newpage

\section*{Kurzfassung} \label{sec:kurzfassung}
% Start cursive block
\begin{itshape}
Deutsche Kurzfassung der Arbeit.
\end{itshape}


% Add Abstract chapter to the same page as Kurzfassung
{\let\clearpage\relax\section*{Abstract}}
% Start cursive block
\begin{itshape}
English abstract of the thesis.
\end{itshape}

% Manually add Chapter names to the header (per default not included by Latex)
\thispagestyle{fancy_beginning}
\renewcommand{\chaptermark}[1]{\markboth{#1}{}}
\chaptermark{Abstract}

%Insert empty page after abstract page
\newpage\null\thispagestyle{empty}\newpage



% Add Table of Contents 
\setlength{\cftbeforechapskip}{0.2cm}			% space before chapters
\renewcommand\cftchapafterpnum{\vskip-0.4cm}	% space after chapters
\renewcommand\cftsecafterpnum{\vskip-0.4cm}		% space after sections
\renewcommand\cftsubsecafterpnum{\vskip-0.4cm}	% space after subsections	
\renewcommand{\cftchapfont}{\normalfont}		% set normal font, not bold
%\titleformat{\section}{\fontsize{14}{19}}	 	% ?
 
\renewcommand*\contentsname{Table of Contents}  % Change the defaulf name "Contents" to "Table of Contents"  
\tableofcontents								% Generate Table of Contents

% Manually add Chapter names to the header (per default not included by Latex)
\thispagestyle{fancy_beginning}
\renewcommand{\chaptermark}[1]{\markboth{#1}{}}
\chaptermark{Table of Contents}

%Insert empty page after table of contents
\newpage\null\thispagestyle{empty}\newpage

% Add List of Figures  
\renewcommand{\cftfigfont}{Figure }				% Add "Figure" to the LOF
\renewcommand\cftfigaftersnum{:} 				% Put ":" after the figure num 
\setlength{\cftfigindent}{0pt}					% Remove left indent 
\setlength\cftbeforefigskip{-0.2cm}				% Space between figures 
\addtocontents{lof}{\vspace*{10pt}}			    % Set distance between LOF title and list
\listoffigures
%\setlength{\cftbeforeloftitleskip}{0cm}		% ? 
%\setlength\cftafterloftitleskip{-2cm}


% Manually add Chapter names to the header (per default not included by Latex)
\thispagestyle{fancy_beginning}
\renewcommand{\chaptermark}[1]{\markboth{#1}{}}
\chaptermark{List of Figures}

%Insert empty page after table the list of figures
\newpage\null\thispagestyle{empty}\newpage

% Add List of Tables
\renewcommand{\cfttabfont}{Table }				% Add "Figure" to the LOF
\renewcommand\cfttabaftersnum{:} 				% Put ":" after the figure num 
\setlength{\cfttabindent}{0pt}					% Remove left indent
\setlength\cftbeforetabskip{-0.2cm}				% Space between tables
\addtocontents{lot}{\vspace*{10pt}}			    % Set distance between LOT title and list
\listoftables{}

% Manually add Chapter names to the header (per default not included by Latex)
\thispagestyle{fancy_beginning}
\renewcommand{\chaptermark}[1]{\markboth{#1}{}}
\chaptermark{List of Tables}

%Insert empty page after the list of tables 
\newpage\null\thispagestyle{empty}\newpage

% Print table of acronyms
\printglossary[type=\acronymtype, title={Table of Acronyms}, nonumberlist, style=longheader]

% Manually add Chapter names to the header (per default not included by Latex)
\thispagestyle{fancy_beginning}
\renewcommand{\chaptermark}[1]{\markboth{#1}{}}
\chaptermark{Table of Acronyms}

%Insert empty page after table of acronyms
\newpage\null\thispagestyle{empty}

% Add Table of Symbols
%\chapter*{Table of Symbols} \label{sec:symbols}
\mbox{}
\nomwithdim{B}{\( F \)}{Force}{L}{$N$}
\nomwithdim{B}{\( g \)}{Gravitational acceleration}{}{$m/s^2$}
\nomwithdim{G}{\( \alpha \)}{Angle of attack}{}{$rad$}
\nomwithdim{G}{\( \zeta \)}{Damping of a linear second order system}{}{--}
\nomwithdim{S}{\( m \)}{Variable related to pitch moment}{}{}
\nomwithdim{S}{\( W \)}{Wind}{}{}
\renewcommand{\nomname}{Table of Symbols}
\printnomenclature[6em]

% Manually add Chapter names to the header (per default not included by Latex)
\thispagestyle{fancy_beginning}
\renewcommand{\chaptermark}[1]{\markboth{#1}{}}
\chaptermark{Table of Symbols}

%Insert empty page after table of symbols
\newpage\null\thispagestyle{empty}\newpage

% Beginning of the core text 
%%%%%%%%%%%%%%%%%%%%%%%%%%%%%%%%%%%%%%%%%%%%%%%%%%%%%%%%%%%%%%%%%%%%%%%

% Activate arabic page numbering from here on
\pagenumbering{arabic}

% Delete the word "Chapter" leaving only the number of the chapter
\titleformat{\chapter}
{\normalfont\chapfnt\bfseries}{\thechapter}{1em}{\chapfnt} 	% set font size
\titlespacing*{\chapter}{0cm}{-7mm}{0cm}					% set spacing

% Set subsection font size (11pt) and spacing
\newcommand{\ssecfnt}{\fontsize{11}{14}}
\titleformat{\subsection}
{\normalfont\ssecfnt\bfseries}{\thesubsection}{1em}{}
\titlespacing*{\subsection}{0cm}{-0cm}{0cm}					% set spacing

% Activate default fancy design (chapter number, chapter)
\pagestyle{fancy} 
\renewcommand{\chaptermark}[1]{\markboth{\ \thechapter \ #1}{}}

% Change separator in the figure and table captures from "." to "-"
\renewcommand{\thefigure}{\thechapter-\arabic{figure}}
\renewcommand{\thetable}{\thechapter-\arabic{table}}

\chapter{Introduction} \label{cha:introduction}

% Manually add Chapter names to the header (per default not included by Latex)
\thispagestyle{fancy}
\chaptermark{Introduction}

The Introduction shall give a short overview over the chapters. Here, this section is used to explain the formatting of thesis papers at FSD. 

\section{General} \label{sec:general}

This template works best, when using “English” as Language in Word, since otherwise some references to styles might not be processed correctly. 
First of all, enter your name and the title of the thesis in the word ‘file’ tab (see Figure \ref{fig:word}). By doing this, all occurrences of the name and tile in this document will be replaced accordingly when refreshing the document.

\begin{figure}[H]
  \centering 
  \includegraphics[width=1 \linewidth]{figs/fig_word.pdf}
  % Figure description is left aligned)
  \captionsetup{justification=raggedright, singlelinecheck=off, font=bf, belowskip=-0.5cm}
  \caption[Titel und Autor als Dokumenteigenschaften]{Titel und Autor als Dokumenteigenschaften}
  \label{fig:word}
\end{figure} 

Enter the German title of the thesis on the first page, and choose the appropriate type from the drop-down menu. Matriculation number and supervisor have to be filled in, and the date when the thesis is handed in can be chosen via an interactive element.

In the text of the Statutory Declaration, the type of work and hand-in date have to be filled in.

On the following page, a German and an English Abstract have to be included, followed by a table of contents for all headings Furthermore, tables for figures and tables have to be included, and the explanation of abbreviations and the nomenclature have to be included in the form of tables.

The text of all chapter is to be written in justified alignment, with a font size of 11 and in ‘Arial’. The font color is black, and the line spacing is to be set to 1.15 (the “Normal” Style in this template). Names (file names, name of bus objects etc.) are to be written in italics (“Emphasis” Style in this template).

Headings are numerated with Arabic characters and are again to be written in Arial. The font sizes are as follows (Microsoft standard with black font and written in Arial):
\begin{description}[labelindent=1cm, topsep=0pt, noitemsep]
	%\setlength{\itemsep}{-10pt}
	\item First level = Chapters: 14, bold (Style “Heading 1”)
	\item Second level: 13, bold (Style “Heading 2”)
	\item Third level: 11, bold (Style “Heading 3”)
	\item Fourth level: 11, bold, italic (Style “Heading 4”)
	\item Fifth level: 11, non-bold, non-italic (Style “Heading 5”)
\end{description}
Every chapter (style ``Heading 1'') automatically starts on a new page. Abstracts, Contents and Lists, the first chapter and the bibliography have to start on an uneven page ($\rightarrow$ ``section break (odd page)'') 

References to sections, figures, tables and equations are to be done as cross references. Examples for references can be found in section \textbf{Fehler! Verweisquelle konnte nicht gefunden werden..}

The bibliography is to be included according to the IEEE style, and references to sources are preferably to be included in square brackets \cite{Handbook}, \cite{Cooper}. The respective bibliography item has to describe the source clearly without ambiguity and citations from sources have to be marked with quotation marks.

Headers and footers are different for even and uneven pages. The header contains the FSD and TUM logo, the current heading number and heading text for the first level. The footer contains the topic, name of the author and page number.

The page numbers of the chapters (introduction until conclusion) are numerated in Arabic characters. The rest are numerated with Latin characters, where the enumeration starts anew from 1 with the Bibliography.


\section{Figures, tables and equations} \label{sec:figs_tabs_eqs}

Figures, tables and equations are to be centered in a separate line. The file format for figures should be *.emf (enhanced windows metafile) and captions for figures and tables are to be inserted below and centered. The font of the caption is Arial (bold) and the font size is 9. Equations are centered with the aid of invisible tables: The middle column contains the equation, the right column the equation number. The font of the equation number is Arial (bold) and the font-size is 9. In every caption, the chapter number is to be included, and with the beginning of every new chapter, the enumeration starts again at 1. Due to the complexity of this layout element, the best way to create new equations is by copying and pasting an existing equation. The equation then first is displayed to be the same as the one copied from. This is updated when performing printing or print preview, or by selecting the equation number and pressing the F9-key.

\begin{figure}[H]
  \centering 
  \includegraphics[width=1 \linewidth]{figs/fig_sample_1.pdf}
  % Figure description is left aligned
  \captionsetup{justification=raggedright, singlelinecheck=off, font=bf, belowskip=-0.5cm}
  \caption[Sample figure 1]{Sample figure 1}
  \label{fig:sample_1}
\end{figure} 

Here, it is referenced to Figure \ref{fig:sample_1} (cross-reference, Reference Type: Figure, Insert reference to: only label and number)

% Add table ([...ex] adds some extra vertical spacing after corresponding row)
\begin{table}[H]
  \centering
  \begin{tabular}{c c} 
    Column 1 & Column 2  \\ [0.5ex] 
    Content & Content \\ 
    Content & Content \\
    Content & Content \\
    Content & Content \\ [1ex] 
  \end{tabular}
  \captionsetup{font=bf, belowskip=-0.5cm}
  \caption[Sample table 1]{Sample table 1}
  \label{table:tab_1}
\end{table}

This is a reference to Table \ref{table:tab_1}. (Cross-reference, Reference Type: Figure, Insert reference to: only label and number)

% Insert equation
\begin{equation}
	R_m = 10 + \Big(\frac{1}{-8.3^{n-1}}\Big) \displaystyle\prod_{i=1}^{n} (R_i-10)
	\label{eq:eq_1}
\end{equation}

Reference to equation \ref{eq:eq_1}. (Cross-reference, Reference Type: Equation, Insert reference to: entire caption) 
The following section 2 contains further example text, figures, tables and equations. 
In section 3 some concluding remarks can be found. Here, we use the acronym \acrfull{ADF}. The second acronym is \acrfull{ADI}. After the first usage of the 1st acronym, only the short version is used: \acrshort{ADF}.

\chapter{Section 2} \label{sec:section_2}

% Manually add Chapter names to the header (per default not included by Latex)
\thispagestyle{fancy}
\chaptermark{Section 2}

Lorem ipsum dolor sit amet, consetetur sadipscing elitr, sed diam nonumy eirmod tempor invidunt ut labore et dolore magna aliquyam erat, sed diam voluptua. At vero eos et accusam et justo duo dolores et ea rebum. Stet clita kasd gubergren, no sea takimata sanctus est Lorem ipsum dolor sit amet. Lorem ipsum dolor sit amet, consetetur sadipscing elitr, sed diam nonumy eirmod tempor invidunt ut labore et dolore magna aliquyam erat, sed diam voluptua. At vero eos et accusam et justo duo dolores et ea rebum. Stet clita kasd gubergren, no sea takimata sanctus est Lorem ipsum dolor sit amet. Lorem ipsum dolor sit amet, consetetur sadipscing elitr, sed diam nonumy eirmod tempor invidunt ut labore et dolore magna aliquyam erat, sed diam voluptua. At vero eos et accusam et justo duo dolores et ea rebum. Stet clita kasd gubergren, no sea takimata sanctus est Lorem ipsum dolor sit amet.
Duis autem vel eum iriure dolor in hendrerit in vulputate velit esse molestie consequat, vel illum dolore eu feugiat nulla facilisis at vero eros et accumsan et iusto odio dignissim qui blandit praesent luptatum zzril delenit augue duis dolore te feugait nulla facilisi. Lorem ipsum dolor sit amet, consectetuer adipiscing elit, sed diam nonummy nibh euismod tincidunt ut laoreet dolore magna aliquam erat volutpat.


\section{Subsection 2.1} \label{sec:subsection_21}

\subsection{Subsection 2.1.1} \label{sec:subsection_211}

Ut wisi enim ad minim veniam, quis nostrud exerci tation ullamcorper suscipit lobortis nisl ut aliquip ex ea commodo consequat. Duis autem vel eum iriure dolor in hendrerit in vulputate velit esse molestie consequat, vel illum dolore eu feugiat nulla facilisis at vero eros et accumsan et iusto odio dignissim qui blandit praesent luptatum zzril delenit augue duis dolore te feugait nulla facilisi.

\begin{figure}[H]
  \centering 
  \includegraphics[width=1 \linewidth]{figs/fig_sample_1.pdf}
  % Figure description is left aligned
  \captionsetup{justification=raggedright, singlelinecheck=off, font=bf, belowskip=-0.5cm}
  \caption[Sample figure 2]{Sample figure 2}
  \label{fig:sample_2}
\end{figure} 

Ut wisi enim ad minim veniam, quis nostrud exerci tation ullamcorper suscipit lobortis nisl ut aliquip ex ea commodo consequat. Duis autem vel eum iriure dolor in hendrerit in vulputate velit esse molestie consequat, vel illum dolore eu feugiat nulla facilisis at vero eros et accumsan et iusto odio dignissim qui blandit praesent luptatum zzril delenit augue duis dolore te feugait nulla facilisi.

Duis autem vel eum iriure dolor in hendrerit in vulputate velit esse molestie consequat, vel illum dolore eu feugiat nulla facilisis at vero eros et accumsan et iusto odio dignissim qui blandit praesent luptatum zzril delenit augue duis dolore te feugait nulla facilisi. Lorem ipsum dolor sit amet, consectetuer adipiscing elit, sed diam nonummy nibh euismod tincidunt ut laoreet dolore magna aliquam erat volutpat.

% Add table ([...ex] adds some extra vertical spacing after corresponding row)
\begin{table}[H]
  \centering
  % Spacing added between columns
  \begin{tabular}{c c@{\hskip 1.5in} c} 
    & Column 1 & Column 2  \\ [0.5ex] 
    Line 1 & Content & Content \\ 
    Line 2 & Content & Content \\
    Line 3 & Content & Content \\ [1ex] 
  \end{tabular}
  \captionsetup{font=bf, belowskip=-0.5cm}
  \caption[Sample table 2]{Sample table 2}
  \label{table:tab_2}
  
\end{table}
Duis autem vel eum iriure dolor in hendrerit in vulputate velit esse molestie consequat, vel illum dolore eu feugiat nulla facilisis at vero eros et accumsan et iusto odio dignissim qui blandit praesent luptatum zzril delenit augue duis dolore te feugait nulla facilisi. Lorem ipsum dolor sit amet, consectetuer adipiscing elit, sed diam nonummy nibh euismod tincidunt ut laoreet dolore magna aliquam erat volutpat.

\subsection{Subsection 2.1.2} \label{sec:subsection_212}

Ut wisi enim ad minim veniam, quis nostrud exerci tation ullamcorper suscipit lobortis nisl ut aliquip ex ea commodo consequat. Duis autem vel eum iriure dolor in hendrerit in vulputate velit esse molestie consequat, vel illum dolore eu feugiat nulla facilisis at vero eros et accumsan et iusto odio dignissim qui blandit praesent luptatum zzril delenit augue duis dolore te feugait nulla facilisi.

% Add table ([...ex] adds some extra vertical spacing after corresponding row)
\begin{table}[H]
  \centering
  % Spacing added between columns
  \begin{tabular}{c c@{\hskip 1in} c} 
    & Column 1 & Column 2  \\ [0.5ex] 
    Line 1 & Content & Content \\ 
    Line 2 & Content & Content \\
    Line 3 & Content & Content \\
    Line 4 & Content & Content \\ [1ex] 
  \end{tabular}
  \captionsetup{font=bf, belowskip=-0.5cm}
  \caption[Sample table 3]{Sample table 3}
  \label{table:tab_3}
\end{table}

Nam liber tempor cum soluta nobis eleifend option congue nihil imperdiet doming id quod mazim placerat facer possim assum. Lorem ipsum dolor sit amet, consectetuer adipiscing elit, sed diam nonummy nibh euismod tincidunt ut laoreet dolore magna aliquam erat volutpat. Ut wisi enim ad minim veniam, quis nostrud exerci tation ullamcorper suscipit lobortis nisl ut aliquip ex ea commodo consequat.

% Insert equation
\begin{equation}
	R_m = 10 + \Big(\frac{1}{-8.3^{n-1}}\Big) \displaystyle\prod_{i=1}^{n} (R_i-10)
	\label{eq:eq_2}
\end{equation}

Duis autem vel eum iriure dolor in hendrerit in vulputate velit esse molestie consequat, vel illum dolore eu feugiat nulla facilisis at vero eros et accumsan et iusto odio dignissim qui blandit praesent luptatum zzril delenit augue duis dolore te feugait nulla facilisi. Lorem ipsum dolor sit amet, consectetuer adipiscing elit, sed diam nonummy nibh euismod tincidunt ut laoreet dolore magna aliquam erat volutpat. 

% Insert equation
\begin{equation}
	R_m = 10 + \Big(\frac{1}{-8.3^{n-1}}\Big) \displaystyle\prod_{i=1}^{n} (R_i-10)
	\label{eq:eq_3}
\end{equation}

Ut wisi enim ad minim veniam, quis nostrud exerci tation ullamcorper suscipit lobortis nisl ut aliquip ex ea commodo consequat. Duis autem vel eum iriure dolor in hendrerit in vulputate velit esse molestie consequat, vel illum dolore eu feugiat nulla facilisis at vero eros et accumsan et iusto odio dignissim qui blandit praesent luptatum zzril delenit augue duis dolore te feugait nulla facilisi.
Nam liber tempor cum soluta nobis eleifend option congue nihil imperdiet doming id quod mazim placerat facer possim assum. Lorem ipsum dolor sit amet, consectetuer adipiscing elit, sed diam nonummy nibh euismod tincidunt ut laoreet dolore magna aliquam erat volutpat. Ut wisi enim ad minim veniam, quis nostrud exerci tation ullamcorper suscipit lobortis nisl ut aliquip ex ea commodo consequat.

\section{Subsection 2.2} \label{sec:subsection_22}

Lorem ipsum dolor sit amet, consetetur sadipscing elitr, sed diam nonumy eirmod tempor invidunt ut labore et dolore magna aliquyam erat, sed diam voluptua. At vero eos et accusam et justo duo dolores et ea rebum. Stet clita kasd gubergren, no sea takimata sanctus est Lorem ipsum dolor sit amet. Lorem ipsum dolor sit amet, consetetur sadipscing elitr, sed diam nonumy eirmod tempor invidunt ut labore et dolore magna aliquyam erat, sed diam voluptua. At vero eos et accusam et justo duo dolores et ea rebum. Stet clita kasd gubergren, no sea takimata sanctus est Lorem ipsum dolor sit amet. Lorem ipsum dolor sit amet, consetetur sadipscing elitr, sed diam nonumy eirmod tempor invidunt ut labore et dolore magna aliquyam erat, sed diam voluptua. At vero eos et accusam et justo duo dolores et ea rebum. Stet clita kasd gubergren, no sea takimata sanctus est Lorem ipsum dolor sit amet.
Duis autem vel eum iriure dolor in hendrerit in vulputate velit esse molestie consequat, vel illum dolore eu feugiat nulla facilisis at vero eros et accumsan et iusto odio dignissim qui blandit praesent luptatum zzril delenit augue duis dolore te feugait nulla facilisi. Lorem ipsum dolor sit amet, consectetuer adipiscing elit, sed diam nonummy nibh euismod tincidunt ut laoreet dolore magna aliquam erat volutpat.
Lorem ipsum dolor sit amet, consetetur sadipscing elitr, sed diam nonumy eirmod tempor invidunt ut labore et dolore magna aliquyam erat, sed diam voluptua. At vero eos et accusam et justo duo dolores et ea rebum. Duis autem vel eum iriure dolor in hendrerit in vulputate velit esse molestie consequat, vel illum dolore eu feugiat nulla facilisis at vero eros et accumsan et iusto odio dignissim qui blandit praesent luptatum zzril delenit augue duis dolore te feugait nulla facilisi. Lorem ipsum dolor sit amet, consectetuer adipiscing elit, sed diam nonummy nibh euismod tincidunt ut laoreet dolore magna aliquam erat volutpat.
At vero eos et accusam et justo duo dolores et ea rebum. Stet clita kasd gubergren, no sea takimata sanctus est Lorem ipsum dolor sit amet. Lorem ipsum dolor sit amet, consetetur sadipscing elitr, sed diam nonumy eirmod tempor invidunt ut labore et dolore magna aliquyam erat, sed diam voluptua. Ut wisi enim ad minim veniam, quis nostrud exerci tation ullamcorper suscipit lobortis nisl ut aliquip ex ea commodo consequat. Duis autem vel eum iriure dolor in hendrerit in vulputate velit esse molestie consequat, vel illum dolore eu feugiat nulla facilisis at vero eros et accumsan et iusto odio dignissim qui blandit praesent luptatum zzril delenit augue duis dolore te feugait nulla facilisi.
Nam liber tempor cum soluta nobis eleifend option congue nihil imperdiet doming id quod mazim placerat facer possim assum. Lorem ipsum dolor sit amet, consectetuer adipiscing elit, sed diam nonummy nibh euismod tincidunt ut laoreet dolore magna aliquam erat volutpat.


\chapter{Conclusions and perspective} \label{sec:conclusions}

In this chapter, concluding remarks should be mentioned and an outlook to future work shall be given.

% Manually add Chapter names to the header (per default not included by Latex)
\thispagestyle{fancy}
\chaptermark{Conclusions and Perspective}

\newpage

% Activate and reset roman page numbering
\pagenumbering{roman}
\setcounter{page}{1}

% Add references
\renewcommand\bibname{References}		% Rename Bibliography to References
\bibliographystyle{plain}
\bibliography{Literature}

% Manually add Chapter names to the header (per default not included by Latex)
\thispagestyle{fancy_beginning}
\renewcommand{\chaptermark}[1]{\markboth{#1}{}}
\chaptermark{References}

\chapter*{Appendix} \label{sec:appendix}
% Add Appendix chapter manually to the Table of Contents because Latex doesn't automatically include nonnumerated (...*) chapters/sections to the TOC
\addcontentsline{toc}{chapter}{Appendix}

% Manually add Chapter names to the header (per default not included by Latex)
\thispagestyle{fancy_beginning}
\renewcommand{\chaptermark}[1]{\markboth{\MakeUppercase{#1}}{}}
\chaptermark{Appendix}

% End of the document
\end{document}